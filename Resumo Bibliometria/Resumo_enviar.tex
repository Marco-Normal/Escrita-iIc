\documentclass[12pt,a4paper]{article}
\usepackage[utf8]{inputenc}
\usepackage[brazil]{babel}
\usepackage{indentfirst}
\usepackage{chemformula}
\begin{document}
\section{Motivação Inicial}

Com o enorme avanço tecnológico que se vem ocorrendo nos últimos anos, a quantidade de poluição vem crescendo exponencialmente, prejudicando ecossistemas, o equilíbrio natural e até mesmo a vida humana. \par

A poluição do ar é um dos problemas ambientais mais graves que o mundo enfrenta atualmente. Ela é resultado de uma série de fatores, como a queima de combustíveis fósseis, emissões de gases tóxicos e fumaça liberada por indústrias, queima de lixo, entre outros. \par

Uma grande consequência desse aumento da poluição é a maior liberação de gases de efeito estufa, como CFC's, aerosóis e em principal o \ch{CO2}. Sendo ele um dos poluentes mais liberados pela humanidade.
Diante dessa problematica e alinhada com os objetivos da ONU de desenvolvimento sustentável, se tem a necessidade de um monitoramento mais rigoroso da liberação desses gases. \par

Nos últimos anos, um novo método de análise surgiu que vem se provando muito eficiente e sendo usado, principalmente para a medição de CO2, que é a utilização de satélites.
Essa tecnologia vem se mostrando muito promissora, principalmente devido ao fato que medições comum, \textit{in situ}, são bastante ineficientes quando precisa se medir em grande escala e outros métodos não apresentam uma boa eficiência. \par

Com a utilização de satélites, por sua vez, devido sua órbita, consegue-se medir em enorme escala, e a depender, de como suas revoluções acontecem ao redor da Terra, o mesmo local diversas vezes.
Essa nova tecnologia aparenta bastante promissora, porém, como qualquer outra, apresenta dificuldades quanto a sua implementação e medições, fazendo com que seu uso ainda seja relativamente novo e de difícil implementação.

\section{Medição por satélites}

A medição de co2 por satélites é um método bastante recente que toma vantagem das diferentes bandas de absorção do carbono. Um raio luminoso é disparado do satélite onde ele atravessa a atmosfera, sendo absorvido pelos átomos dos gases presentes. Então, com sua eventual refração, consegue-se medir a composição, aproximada de X\ch{CO2}, que é a coluna de \ch{CO2} presente.
Como esse método é extremamente dependente da luz, qualquer outra refração ou absorção causada, seja por nuvens, poluentes, aerosóis, causa uma grande imprecisão na leitura. \par

Um dos grandes problemas desse método é a sua precisão. Espécies como o oxigênio, que possuem bandas próximas do \ch{CO2}, assim como nuvens, aerossóis e até mesmo raios solares possuem uma grande interferência na precisão da medição.
Além disso, diferentes satélites tem métodos de varredura completamente distintos, causando fluxos de dados com diferenças significativas, que faz com que a comparação entre eles seja difícil. \par

\section{Algoritmos de medição}

Fazer a medição é apenas uma parte do trabalho. Uma questão bastante complicada, muitas vezes a causadora grande de problemas são os algoritmos utilizados.
Como já falado, os principais satélites utilizam de diferentes meios de medição, logo, para cada satélite foi-se desenvolvido um algoritmo diferente. \par

Para o OCO-2, por exemplo, a Nasa utiliza um algoritmo chamado \textit{ACOS}, que utiliza de um método de \textit{Maximum Likelihood Estimation} para fazer a medição. Esse algoritmo possui uma falha, que ele não consegue lidar com nuvens muito bem, então, em dias nublados, as medições não serão as mais acuradas. \par

Analisando o GOSAT, utiliza um algoritmo chamado \textit{NIES}, que utiliza do método de \textit{Maximum Likelihood Estimation}. Esse método também inclui um algoritmo de detecção de nuvens e também utiliza dados das bandas espectrais do metano e oxigênio para optimizar melhor seus cálculos. \par

Além disso, existem outros métodos, desenvolvidos para esses satélites, mas por equipes independentes, como o método \textit{UoL-FP}, desenvolvido pela universidade de Leicester para o satélite OCO-2. Esse modelo é bastante robusto, simulando uma atmosfera e fazendo tratamento para aerossóis e nuvens durante os cálculos desenvolvidos. \par

Pode ser observado uma enorme variedade de algoritmos, cada um com suas variáveis e métodos de cálculo, fazendo com que a comparação entre eles seja bastante complicada. Com isso, existe uma grande dificuldade de validação cruzada entre ele, dificultando a acurácia e promovendo erros. \par

\section{Objetivos e Metodologia}

Os objetivos que temos para esse trabalho são
\begin{itemize}
	\item Fazer uma análise bibliométrica dos artigos publicados sobre métodos de tratamento de dados de satélites;
	\item fazer uma leitura e análise dos métodos estatísticos utilizados por esses algoritmos;
	\item Realizar uma síntese e comparação entre eles;
	\item Realizar uma possível crítica aos principais métodos
\end{itemize}

Portanto, uma grande parte do trabalho é a realização de uma análise bibliométrica detalhada, sendo necessário, portanto, a escolha adequada do satélite para qual será analisado. As escolhas principais ficam entre o OCO-2 e o GOSAT, ambos já pela enorme quantidade de métodos desenvolvidos e suas enormes importâncias para a comunidade científica. \par

\subsection{Análise Bibliométrica}
Faz-se necessário também, a escolha correta das palavras-chave para a busca dos artigos de interesse, já que uma escolha incorreta levaria a uma gama de artigos não muito útil. A pesquisa preliminar realizada na "Web of Science" foram:

(Measurement* OR quantifying OR prediction*) AND ((atmospheric* AND (carbon OR CO2 OR XCO2 )) AND
(stock* OR pool* OR emission* OR sink* OR budget OR  source*)) AND satellite* \\

Onde foram buscados tanto nos resumos, títulos e palavras-chave. Essa pesquisa gerou um total de 783 artigos. Esse número é extremamente significativo, precisando de uma filtragem maior, já que uma análise completa seria algo astronômico. \par

Foi proposta então uma limitação, onde só seriam analisados artigos que falassem de um dos satélites
já citados, o \textit{OCO-2} ou o \textit{GOSAT}. Limitando para o \textit{GOSAT}, utilizando as
palavras-chave: \\
(Measurement* OR quantifying OR prediction*) AND ((atmospheric* AND (carbon OR CO2 OR
XCO2 )) AND (stock* OR pool* OR emission* OR sink* OR budget OR  source*)) AND satellite* AND GOSAT.\\
Gerando um resultado de 83 artigos. Um número consideravelmente menor e já palpável quanto sua possibilidade.  \par
Por sua vez, quando limitados para o \textit{OCO-2}, utilizando as palavras-chave: \\
(Measurement* OR
quantifying OR prediction*) AND ((atmospheric* AND (carbon OR CO2 OR XCO2 )) AND (stock* OR pool* OR
emission* OR sink* OR budget OR  source*)) AND satellite* AND OCO-2.\\
Gerando um resultado de 64 artigos. Sendo um número bem menor e mais possível de ser analisado. \par
Uma das maiores dificuldades que vem acontecendo é a apreensão quanto a escolha das palavras-chave. Enquanto não queremos limitar demais a pesquisa a ser feita, existe um limite quanto a quantidade de artigos que podem ser analisados. Enquanto a quantia de 93 é relativamente palpável, ainda possuímos algumas incertezas quanto a:
\begin{itemize}
    \item Essas palavras-chave são suficientes? Estamos deixando uma de lado que é muito importante?
    \item Essas palavras-chave são muito limitantes? Estamos deixando de fora artigos que são importantes?
    \item A limitação quanto aos satélites é boa?
    \item Esses satélites são os mais importantes? Existe outro que mostra mais relevante que esses?
\end{itemize}
\section{Resultados Esperados}

Se espera que com esse trabalho, seja possível fazer uma análise mais detalhada dos métodos utilizados para o tratamento de dados de satélites, assim como uma comparação entre eles, fazendo com que seja possível uma melhor compreensão dos métodos utilizados e uma crítica aos mesmos. \par

Ademais, se espera que com esse trabalho se crie uma base sólida para a realização de pesquisa futuras e com uma possibilidade de uma entrada na gama de criação de ideias ou melhorias aos atuais algoritmos empregados.

\end{document}
