\documentclass{article}[12pt]
\usepackage[utf8]{inputenc}
\usepackage[brazil]{babel}
\usepackage{xcolor,amsmath,adjustbox,makecell}
\usepackage{hyperref}
\usepackage[num]{abntex2cite}
\date{}
\author{}
\citebrackets[]

\begin{document}
\section{Introdução}
Sobre uma análise preliminar, olhando o que está sendo pesquisado na área de bibliometria e
sustentabilidade, não muito está acontecendo, muito menos quando se fala na questão de estatística.
Lendo alguns artigos sobre pesquisas bibliometrias em sustentabilidade, a grande maioria dos
resultados mostra que existe um crescimento, mas algumas área ainda não mostram a importância desse
assunto. Dentre elas a área de engenharia ecológica, que basicamente está mais focada em um único
assunto, com poucas publicações saindo de uma mesma norma. \par

Fazendo uma análise na questão de bibliometria de estatística, não se tem, aparentemente, um artigo
que fez um enfoque entre essas áreas, sustentabilidade e estatística, então ainda é algo
aparentemente novo, quanto a isso. \par

Interessantemente, os principais tópicos que estão sendo pesquisados na área de sustentabilidade
são, desenvolvimento sustentável, sustentabilidade ambiental, sustentabilidade urbana, pegada
ecológica, mudanças climáticas \cite{Ellili2023}. Olhando por cima, parece que talvez o mais fácil
de se achar metodologias estatísticas seja na parte de mudanças climáticas, quanto sua
identificação e pegada ecológica, quanto a sua mensuração, mas claro que é válido um olhar mais
critico e uma análise mais profunda. \par %Farei isso ao longo do tempo, isso é só um olhar por cima

Me deixa um pouco preocupado o fato de que, aparentemente, não se tem uma metodologia estatística
muito bem definida, seja talvez por um receio de se usar ou falta de estudos sobre como ela pode ser
usada, mas claro que posso só estar olhando de forma errada o problema. \par


\section{Sustentabilidade}
Sobre um viés, observando artigos bibliométricos sobre a sustentabilidade, por mais que muitos
tenham sido lançados recentemente, percebe-se uma emergência de palavras chaves e tópicos que estão
sendo pesquisados, que vai evoluindo ao longo do tempo. No começo da desada, o foco era mais em
conscientizar e mostrar a importância, já perto de 2006, mudou para um foco em manuseio sustentável
da água e crescimento econômico. Mais recentemente, estão em optimização dos SGD's \cite{Ellili2023}. \par

É interessante que no meio de tudo ainda, ainda se teve a pandemia do COVID-19, que faz com que
talvez se tenha uma distorção do que realmente se estava pesquisando e quais vão ser a tendencias,
já que agora, imagino eu, se tera uma emergência bastante grande de artigos que vão relacionar esse
tópico com a sustentabilidade. \par

Ainda não consegui ver com muita clareza palavras chaves que usaram para fazer a pesquisa
bibliométrica, mas vou continuar pesquisando mais para ver se consigo achar algo mais concreto. \par

Em questão de palavras chaves, algumas menções vão para
\begin{itemize}
    \item Urban Metabolism
    \item Urban Ecossystem
    \item Sustainability accounting
    \item industrial ecology
    \item industrial symbiosis
    \item industrial metabolism
    \item industrial ecosystem
    \item socio-economic metabolism
    \item life cycle analysis
    \item material flow analysis
    \item life cycle assessment
    \item input-output
\end{itemize}

Na parte de estatisitica eu particularmente não vi nada relacionado com a sustentabilidade, no
quesito de trabalhos feitos, então é algo inedito, ao meu ver. Mas pode ser algo difícil, como já
mencionado. Não se tem tantos trabalhos usando a estatística, mas os que tem são bastante
proveitosos e de qualquer forma acho válido que sejam feitas análises. \par

\section{Resiliência}

Na parte de resiliência, aparentemente se tem uma maior variedade quanto ao uso de métodos
estatisticos. Tanto no uso de teoria da informação e entropia \cite{Ulanowicz2009}, quanto o uso de outros métodos
também. %Pesquisar mais

Em uma outra análise, esse tópico também é extremamente diverso, de forma que dentro da resiliência,
algumas áreas vem utilizando mais, de forma mais abrangente, metricas, como por exemplo Ecological
Resilience \cite{Cushman2019}. \par



%Pesquisar mais no geral. Achando meio intrigante não ter tanta coisa de estatistica. Não sei se é a
%forma como estou pesquisando ou algo assim. Posteriormente vou mudar meu jeito de pesquisa. Acho
%que ficar pesquisando analises já prontas parece um pouco redundante, acaba que todas falam a mesma
%coisa...

%Depois tenho que ver sobre a analise de resiliencia. 

% \section{Análise de Caso. Uso de Teoria da informação para avaliar resiliência}
% No artigo \textbf{\emph{Quantifying sustainability: Resilience, efficiency and the
% return of information theory}} \cite{Ulanowicz2009}, por mais que seja um pouco antigo, serve de
% base para diversos outros, já que serviu de chão para a criação de uma metodologia para quantificar
% sustentabilidade e resiliência dentro de um sistema. \par

\section{Um estudo sobre métodos de medição de CO2 por satélites}
Um técnica interessante e que desconhecia quanto sua existência, era a medição de dióxido de carbono
e outros poluentes por meio de satélites, enquanto não tão nova e já existiram a um certo tempo,
essa técnica vem crescendo em popularidade e estudos vem sendo conduzidos quanto a meios de se
melhorar as medições. \par

A ideia de precisarmos disso é do fato de que, por meio das resoluções da ONU, quanto ao
desenvolvimento sustentavel, se mostra necessário o monitoramento de poluentes e a quantidade de CO2
e outros gases nocivos que se libera por um país. Enquanto se é possível medir por sensores
terrestres, a técnica se mostra custosa e precisa de uma grande área, teoricamente o país todo, ser
monitorado para que realmente seja eficiente, fato claro que não é tão viável. \par

Então, se tem a ideia de usar satélites para que se tenha essa medição. Enquanto sua maioria não
possui a maior das áreas, algumas variando de 2x2 km, até 10x10, o fato de poder ser escaneado o
país todo, e por fato, uma grande área do globo, se mostra vital que seja mais desenvolvida e
implementada. \par

A ideia de funcionamento desses satélites se baseia na medição de espectros de luz, geralmente com
comprimento e onda perto do infravermelho, e com base em cálculos de como a luz varia, conseguem
medir a quantidade de CO2, mais especificadamente, da coluna de CO2 que se tem, ou o XCO2. \par

Um dos grandes problemas disso, é que como se baseia no fato de dependermos da luz, outros
poluentes, nuvens, aerosois, difração e refração da luz podem atrapalhar na nossa medição, de forma
que se precisa leva-los em conta e fazer as devidas correções dos dados que os satélites nos enviam.
\par

Com a adoção dessa tecnologia, vem um problema, que diversos países estão trabalhando em novos
satélites e formas de calcular essa medição, mas ainda não é padronizado, de forma
que se formos fazer uma comparação entre o satélite da Nasa (OCO-3) e o japonês (GOSAT-2), se tem
uma variação quanto a medição, de forma que se dificulta a comparação dos dados. \par

De acordo com \citeonline{Yue2016}, têm-se 9 algoritmos distintos, até 2016, onde se calcula o XCO2
baseado nos dados que os satélites nos enviam.
\subsection{Algoritmos}
Todos os algoritmos são descritos de forma similar ao escrito no artigo \cite{Yue2016}, onde
posteriormente escreverei de forma mais detalhada sobre cada um
\subsubsection{WFE-DOAS}
Esse método se baseia num ajuste sem restrição por mínimos quadrados, desenvolvido especificadamente
para o equipamento no satélite Evisat-1. Para o ajuste linear foi feito a utilização de funções de
pesos para a coluna do peso de gás, um para a temperatura e um polinômio de um baixo grau. O
logaritmo de uma função linearizada de um modelo de  transferência de radiativa, mais um polinômio
de baixo grau é ajustado para a radiância normalizada do sol. \par

Esse modelo foi melhorado usando os Fatores-M, que são fatores multiplicativos relacionados à
calibração radiometrica absoluta. Ademais, esse modelo foi melhorado considerando o perfil vertical
do aerosol para a transferência radiativa e por meio de detectores de nuvens. \par

\subsection{BESD} 

Esse algoritmo utiliza de melhores estimadores e do WFM-DOAS. Para um outro satélite, existe também
o BESD/C, que é similar ao BESD, mudando um pouco quanto ao vetor de estado. O BESD/C é baseado
também na técnica de melhor estimador e usa informações a priori para que aja uma certa restrição.
Ademais, para melhorar mais a velocidade de computação e diminuir os erros sistemáticos de XCO2 e
XCH4, um método de parametrização de erro foi criado como uma função de paramêtros de diversas
entradas críticas, como a profundidade óptica dos Aerosols, altitude cirrus, etc. \par

\subsection{Algoritmo NIES} 

Desenvolvido pelo Japão para seu satélite GOSAT, também é baseado na técnica de melhor estimador. Em
adendo ele também inclui um algoritmo, sem bias, de detecção de nuvens e utiliza bandas espectrais
do carbono, metano e oxigênio para cálculos. Ele foi melhorado mudando-se o dataset de irradiação
solar, melhorando as propriedades ópticas dos aerosols, dentre outros aspectos. \par

\subsection{PPDF}

Esse algoritmo foi desenvolvido implementando a PDF dos caminhos de fótons, que faz com que também
inclua pequenas nuveá de CO2, quando se observa a luz solar no \textit{near-infrared}. A
parametrização do efeito das nuvens é baseado numa analise estatística dos caminhos do fótons
simulada por meio de técnicas de Monte Carlo. Esse método imita o DOAS quando ignoramos modificações
no caminho da luz. \par

\subsection{ACOS}

Esse método foi desenvolvido pela NASA para o satélite OCO. Esse algoritmo também utiliza da técnica
de melhor estimador, onde os parâmetros de entrada de um \textit{Forward Model}
%não sei o que é perguntar dps. 
são optimizados para que deem um espectro simulado que se aproxime do espectro medido, enquanto
simultaneamente são restringidos por informações a prior. \par 

Um dos problemas desse modelo é que ele não consegue lidar com nuvens, ou seja, é um grande problema
para dias em que isso acontece. \par

\subsection{Algoritmo UoL-FP}
Desenvolvido pela universidade de Leicester, esse algoritmo utiliza de um \textit{Forward Model} e
de um método inverso. O \textit{Forward Model} consiste em um modelo de transferência radiativa, um
modelo solar e um modelo de instrumento. O método inverso é baseado na técnica de melhor estimador.
Esse algoritmo foi criado para o satélite OCO-2, ou seja, ele utiliza os mesmos dados que o ACOS, se
diferindo na definição do seu vetor de estado, os valores a priori, covariância e em especial no
tratamento dos aerosols e nuvens cirrus\footnote{Basicamente são as nuvens que parecem algodão,
aquelas que primeiro imaginamos quando falamos de nuvens}. \par

\subsection{RemoTeC}

Esse é um método de \textit{retrieval} que permite que se colete alguns parâmetros efetivos de
aerosols simultaneamente com a coluna de CO2, utilizando a parametrização da quantidade de
partículas, distribuição de alturas, e propriedades microscópicas. Uma das principais qualidades
desse método é conseguir coletar a concentração de gases e as propriedades de dispersão das
partículas atmosféricas usando um modelo muito efetivo de transferência radiativa. \par

\subsection{SECM}

Esse método foi desenvolvido para simular as concentrações ambientes, de \textit{background}, do
CO2, utilizando misturas de razões de concentrações de gases e XCO2. Esse método é uma equação
simples utilizando 17 parâmetros empíricos, latitude e data. Os parâmetros empíricos foram
determinados por meio de encaixe de mínimos quadrados. \par

Esse método, dependendo apenas na data e longitude consegue explicar 94\% da variabilidade na
concentração de CO2 da atmosfera. Esse método pode ser utilizado como um conhecimento a priori para
achar o melhor estimador para outros modelos. \par

\subsection{Método de Câmara}

Esse método foi desenvolvido por \citeonline{Reuter2013} desenvolveu esse método onde ele combina, o
DOAS, BESD, NIES, NIES-PPDF, ACOS, UoL-FP e RemoTeC para criar um novo dataset. O principio básico
se apoia em que cada algoritmo tem suas próprias vantagens e desvantagens, de forma que é
extremamente raro que eles geraram outliers para o mesmo lado, então um "arrumaria" o erro do outro.
\par

A ideia é que o satélite capitaria a radiação solar nas bandas de absorção do CO2 e O2, onde um
modelo de transferência radiativa mais um \textit{forward model} são utilizados para simular as
medidas do satélite para um vetor paramétrico conhecido e um vetor de estado desconhecido. Um método
de inversão então é utilizado para achar o vetor de estado que melhor se ajusta ao resultados
medidos. O vetor de estado é assumido como sendo aquele em que tenha maior probabilidade da
atmosfera estar naquele momento. \par

\section{Dificuldades apresentadas}

Um dos grandes problemas desses algoritmos é a falta de padronização entre eles, de forma que por
exemplo, medidas de XCO2 podem variar entre cada de acordo com a importância dada ao parâmetro de
verticalidade. O algoritmo do OCO calcula por meio da pressão, enquanto o SCIAMACHY calcula
por meio da quantidade de moléculas de ar e pela componente de ar seco. \par

Ademais, até problemas com nuvens e aerossóis, assim como assumir coisas que talvez não sejam
verdade, como o caminho da luz, ja que pode ocorre fenômenos de desvio de caminho, como a refração,
etc, podem gerar grandes erros nos cálculos, se o método for dependente desses dados. \par


Ademais, como a grande maioria das medições são feitas por uso de luz e análise de bandas, tanto do
CO2 tantto do CH4 algumas dificuldades no quesito que outros gases possuem bandas relativamente
perto das bandas de absorção dos gases mencionados, o próprio O2, possui bandas bastante perto do
CO2, menos pronunciadas, mas mesmo assim atrapalham. Nuvens e vapor de água tambem possuem bandas
assim que dificultam a leitura e retorno dos dados. \par

\section{Medição de Carbono}
Enquanto parece um pouco difícil, surpreendetemente, achar algumas bibliometrias sobre medição de
carbono, acho que vem muito do fato de ser relativamente descentralizado, ou talvez por incopetencia
própria mesmo. \par

Mas de qualquer forma, o jeito que aparentam mais se falar disso é utilizando o termo \emph{Carbon
Stocks}, que num termo realmente de bolsa de valores, faz o cálculo do que é lançando para a
atmosfera menos o que é recuperado. Enquanto o artigo de \cite{Yue2016} fala um pouco sobre algumas
estações terrestres de medição, a grande maioria dos artigos falam focam em algumas medições de
forma mais bem localizada. \par

Por exemplo, se tem muitos artigos falando sobre diversos tópicos, desde de sequestro de carbono por
meio de micro-algas a medição de carbono no solo devido a agricultura, o ponto mais importante,
talvez, claro que os outros não devem ser excluídos, é a questão dos \emph{Carbon Stocks}. \par

Um artigo muito interessante, por mais que meio antigo, é \cite{Petrokofsky2012}, onde ele faz uma
revisão sistemática de métodos de cálculo de carbono, onde no apêndice A ele coloca termos de busca
que ele usou para conseguir artigos para fazer sua análise, vale a pena dar uma olhada.
% Palavras chaves de ja vou pensando: Carbon Pools, Carbon Sinks, Carbon Stocks, Carbon
% Sequestration, Carbon Emissions. (Dead Wood Pools? Biomass? Above Ground Biomass? Eddy Covariance?
% soil organic carbon?)
Mais do que apenas medições, muitas estimativas vem apenas de equações e observações feitas
previamente. Quando falamos, por exemplo, da floresta amazionica, é bastante dificil termos uma
exata estimativa de quanto ela esta retirando, ou lançando, então muitas vezes se aproveita de
modelos e equações desenvolvidas para, determinar a bio-massa, determinar quanto essa bio-massa
troca carbono, dentre outras coisas, de forma que deveria ser incluido, na pesquisa
%Provavelmente ja ia, mas bom reforçar
\subsection{Palvras Chaves do Marco}
Em nenhuma ordem em particular, eu diria importante termos
\begin{itemize}
    \item Carbon Stocks
    \item Carbon Pools
    \item Carbon Budget
    \item Carbon Source
    \item Carbon Balance
    \item Carbon Sequestration
    % \item Lidar (Medições usando Lidar)
    \item REDD+
    \item Carbon Emissions
    \item Carbon Sinks
    \item Carbon Flux
    \item ABV
    \item Dead Wood Pools
    \item Peat
    \item Eddy Covariance % Possivelmente mais terrestres
    % \item In situ measurements
    \item Carbon Budget
    \item Near-Infrared
    \item Spectroscopy
\end{itemize}
Agora, para caso seja necessário ja fazer a distinção entre os tipos de medições, podemos usar, para
medição terrestre, essas palavras mais:
\begin{itemize}
    \item \textit{In situ} 
    \item Ground Based
    \item Terrestrial
    \item Tower Based
     % Mais questão de observatorios etc
    \item TCCON (Total Carbon Column Observing Network)
    \item NOAA's 
    \item CONTRAIL
\end{itemize}
Para medição de satélites, podemos já usar essas palavras chaves:
\begin{itemize}
    \item Satellite based
    \item Space based
    \item Above-ground %discutivel se realmente seria útil
    \item Remote Sensing
    % Mais na questão dos satélites
    \item OCO-2
    \item Evisat-1
    \item \dots
\end{itemize}
\bibliography{Bibliografia}
\end{document}