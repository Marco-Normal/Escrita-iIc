\documentclass{article}[12pt]
\usepackage[utf8]{inputenc}
\usepackage[brazil]{babel}
\usepackage{xcolor,amsmath,adjustbox,makecell}
\usepackage{hyperref}
\usepackage[num]{abntex2cite}
\title{\textbf{Modelagem Estatística na área de Sustentabilidade: Uma análise bibliométrica.}}
\date{}
\author{}
\citebrackets[]

\begin{document}
\maketitle

\section{Resumo}

A área de sustentabilidade vem ganhando enorme impulso ao longo dos anos, onde se vê um enorme
crescimento de pesquisas científicas e uma maior preocupação com o impacto humano no meio ambiente.
\par

Nesse contexto, a área de estatística vem ganhando um papel de destaque crescente nessa área,
principalmente pelo seu grande potencial de análise e interpretação de dados. Sendo ainda bastante
recente, tal integração, diversas técnicas diferentes vem sendo aplicadas, em diferentes meios de
atuação.  \par

Dessa forma, durante o projeto será realizado uma revisão e análise bibliométrica da área de
sustentabilidade, sendo feito a criação de um código-fonte com os princípios dessa análise.
Posteriormente, será efetuado uma análise detalhada dos artigos de interesse, buscando identificar
quais técnicas estão sendo as mais utilizadas. Por fim, será elaborado outro código-fonte de livre
acesso, realizando a aplicação das técnicas mais utilizadas em um conjunto de dados de interesse.
\par
% {\color{red} Aguardar para escrever}

\section{Justificativa}

A engenharia e a pesquisa científica sempre possuiu um papel muito importante para o desenvolvimento
humano. Desde a revolução industrial, % até a criação de
inteligências artificiais, os dias essas áreas foram responsáveis por facilitar o trabalho humano.
Entretanto, é importante tentar compreender como esse desenvolvimento está impactando o ambiente em
que vivemos e como o progresso atual impacta nosso planeta. \par

Nesse contexto, a sustentabilidade é definida como o meio a qual consegue-se integrar o
desenvolvimento tecnológico, econômico e social, com a utilização correta e consciente dos recursos
naturais, de modo a construir relações sociais e naturais, sustentáveis, ecológicas e resilientes
para que não se comprometa nem o atual, nem o futuro \cite{UCLA_2021}.   

Na primeira etapa deste projeto, pretende-se fazer uma análise bibliométrica onde será feito um
levantamento bibliográfico detalhado sobre as principais pesquisas na área de sustentabilidade,
assunto este de extrema importância. Já na segunda etapa deste projeto, pretende-se analisar os
artigos mais recentes na área de sustentabilidade detalhadamente, em busca das metodologias
estatísticas empregadas. Isto permitirá uma maior divulgação dos principais resultados já obtidos na
área, além de entender o rumo das pesquisas recentes, quais as áreas que estão tendo um maior foco,
permitindo assim identificar as principais métricas e métodos que estão sendo utilizados, além de
identificar \textit{gaps} e tendências emergentes. \par



%Ademais, entendendo-se as principais métricas e formas de análises utilizadas, consegue-se
%identificar falhas e possíveis  deficiências que possuem, sendo possível corrigi-los e melhorar a
%qualidade das análises e pesquisas futuras. \par



%Uma análise bibliométrica de qualidade permite uma divulgação dos principais resultados e a
%identificação de \textit{gaps} e tendências emergentes. 


A proposta deste projeto vai de encontro com o destacado pelo novo diretor da EESC/USP. Em abril de
2023, Prof. Catalano, no discurso de sua posse como diretor da Escola de Engenharia de São Carlos
(EESC) relembrou a importância da engenharia e seu papel fundamental no progresso humano. Ele
destacou a necessidade de a engenharia estar ligada ao desenvolvimento humano, principalmente em
temas relacionados aos desafios globais, tais como mudanças climáticas, pobreza e proteção do meio
ambiente. Conforme ressaltou em seu discurso ``Só recorrendo à ciência tem sido possível abrir
caminhos para ações mais assertivas. Quanto mais intensas as necessidades de novas descobertas e
soluções, mais se torna importante o campo de trabalho da ciência, da tecnologia e das engenharias.
Trabalho este que a USP faz com maestria. O ensino de engenharia tem que promover não só a
conscientização do problema ambiental, mas a solução do problema'' \cite{Cruz2023} \\ \\



\section{Objetivos}


\subsection{Objetivo Geral}
O objetivo desse projeto é realizar um mapeamento do desenvolvimento cientifico na área da
sustentabilidade, com possível enfoque na(s) sub-área(s) de resiliência, \textit{flips} ecológicos
e/ou quantificação da biodiversidade. Além disso, pretende-se efetuar uma análise detalhada e
rigorosa, visando identificar as principais técnicas e métodos estatísticos que estão sendo
utilizados atualmente. 


\subsection{Objetivos Específicos}

\begin{itemize}
\item Realizar a análise bibliométrica da área de sustentabilidade, com possível enfoque na(s)
sub-área(s) de resiliência, \textit{flips} ecológicos e/ou quantificação da biodiversidade;
%e sinecologia
\item Escrever um código-fonte comentado de livre acesso, com os princípios de análise
bibliométrica;
\item Realizar uma leitura detalhada de todos os artigos mais recentes publicados na(s) sub-área(s)
de interesse, identificando as técnicas estatísticas mais utilizadas atualmente;
\item Aplicar pelo menos duas das técnicas estatísticas em um conjunto de dados, com
disponibilização do código-fonte criado para tal fim.
\end{itemize}

\section{\textbf{Metodologia}}

\subsection{Sustentabilidade}

A área de sustentabilidade compreende uma gama de assuntos, tais como: o estudo da resiliência de um
ecossistema, %\footnote{Para mais informações acesse 
\cite{Dakos2022, Eason2013, Boschetti2019}, análise de \textit{shifts} ecológicos
%\footnote{Para mais informações acesse
\cite{Mayer2006, Clare2017} e quantificação da biodiversidade de um ecossistema
%\footnote{Para maiores informações acesse 
\cite{Sherwin2019, Daly2018, Roswell2021}. \par
%e a sinecologia. {\color{cyan} Acha que precisa de um paragrafo explicando cada, ou apenas a
% explicação acima esta de bom tamanho?  } {\color{red} Acho que pode detalhar um pouco mais. }
O estudo da resiliência de um ecossistema busca quantificar e entender o quanto um ecossistema é
resiliente quanto a pertubações, ou seja, em um acontecimento de alguma interferência, humana ou
não, o quanto ele conseguiria se recuperar e voltar ao seu estado original \cite{Dakos2022}. \par

A análise de \textit{shifts} ecológicos, por modelos, indicadores e análises, estatísticas e
ambientais, quando um ecossistema está sofrendo uma mudança de estado, saindo de seu estado
original, de equilíbrio e indo para outro, que pode ser de equilíbrio ou não \cite{Clare2017}. \par

Já a quantificação da biodiversidade de um ecossistema, por sua vez, através de modelos matemáticos,
estatísticos e computacionais, quantificar a biodiversidade de um ecossistema, ou seja, inferir o
quão diverso ele é,  seja na quantidade de populações, especies, regimes ecológicos. Podendo ser
quantificado até mesmo dentro de uma mesma população o quão diversa ela é \cite{Sherwin2019}. \par

De acordo com a Organização das Nações Unidas (ONU), a sustentabilidade pode ser definida como
"Suprir as necessidades do presente sem comprometer as gerações futuras de suprir as suas próprias
necessidades" \cite{onenvironment1987common}. A partir desta definição, se tem um caminho muito
claro para o futuro desejado do desenvolvimento humano, onde deve-se buscar o equilíbrio entre as
necessidades da atualidade e a saúde do planeta para as próximas gerações. 

\par
Neste sentido, no ano de 2015, a ONU estabeleceu 17 objetivos de desenvolvimento sustentável, que
são uma série de objetivos que devem ser alcançados até 2030 por todos os países membros da ONU
\cite{Agenda2023}. Ao longo destes 15 anos (de 2015 a 2030), todos os países pertencentes à
organização devem adotar políticas alinhadas com esses desenvolvimentos sustentáveis, contribuindo
assim para que se alcance tais objetivos. \par


É de consentimento geral que a sustentabilidade é um dos maiores desafios enfrentados pela
humanidade atualmente. Com o crescimento populacional, o aumento da produção industrial e o aumento
do consumo, a demanda por recursos naturais vem aumentando cada vez mais. Com a proposta da ONU de
desenvolvimento sustentável, se vê crescente a preocupação com esse tema, principalmente com a
implementação de novas políticas ambientais e de desenvolvimento que diversos países vem adotando
\cite{SDG2022},
%{\color{red} como XXX (citar algumas se possível).}
Política nacional de bio-combustíveis \cite{RenovaBio2017}, no Brasil, O plano ABC
\cite{PlanoABC_2016} e o Decreto N° 11.075, este que estabelece medidas para diminuir as mudanças
climaticas observadas \cite{Decreto11075}.Na união europeia, se tem projetos de expandir a
\textit{Natura 2000 network} para 30\% do território \cite{Mller2020}.  Ademais, países como a
França buscam mudar suas fontes de energias para aquelas que impactam menos o meio ambiente, como a
nuclear, solar e eólica \cite{Lebrouhi2022}.
\par


Ainda se observa um enorme espaço disponível para que novas ideias e pesquisas sejam aplicadas nessa
área, de forma a conseguir manter uma balança entre o desenvolvimento humano e a saúde do planeta.  
Neste sentido, a área de engenharia e de estatística, possuem um papel fundamental no
desenvolvimento e no aprimoramento de tecnologias de modo a compreender o problema e buscar soluções
para os desafios globais de sustentabilidade.
\par


Como a área de sustentabilidade é extremamente interdisciplinar e abrangente, se faz necessário uma
análise mais específica sobre o que está sendo feito e quais as tendências que estão sendo seguidas.
O mapeamento geral do desenvolvimento cientifico permitirá que se compreenda tendências e
identifique possíveis \textit{gaps}. A Análise Bibliométrica, que será apresentada na seção
seguinte, é uma ferramenta útil para construir tal mapeamento.\par

\subsection{Estatística}

Estatística é um ramo da matemática que tem ganho cada vez mais importância sua evolução ocorre de
maneira muito rápida. A partir das técnicas desenvolvidas nesta disciplina é possível coletar
informações, organizar, interpretar e  analisar os dados de maneira adequada, o que permite extrair
conclusões corretamente a partir da análise feita. 

Pesquisas são feitas constantemente para tomada de decisões nas mais diversas áreas, a partir da
análise de processos reais. 
%Desde análises simples, como médias e variâncias, a análises mais complexas e rigorosas, como
%testes de hipóteses, explicações de variância e modelos de regressão, a estatística é uma das áreas
%de maiores importâncias para praticamente todas as disciplinas científicas. \par
Na área de sustentabilidade, a estatística vem ganhando uma grande importância, com descobertas e
novas formas de aplicação de modelos estatísticos para a análise em diversos tópicos da área. Em um
trabalho seminal, por exemplo, foi se discutido a teoria da análise de \textit{flips} ecológicos
utilizando a informação de \textit{fisher}, com resultados bastantes positivos, embora algumas
limitações \cite{Mayer2006}. %\par
Com o uso dessa teoria, consegue-se analisar de forma mais rápida e eficiente essas mudanças, para
que identifique se um ecossistema está sofrendo uma mudança de estado e como poderia reverter-se, ou
minimizar esses impactos. Enquanto ainda experimental, artigos mostram bastante promessa para essa
teoria %\footnote{Veja, 
\cite{Karunanithi2008, Rawlings2020, Konig2019}. \par

Ademais, pode-se citar o uso da teoria da informação, que está sendo bastante utilizada, tanto para
quantificar a sensibilidade e robustez de modelos de ecossistemas \cite{Ulanowicz2009}, a
quantificação do quanto há mudança em um sistema, ou seja, suas mudanças em biomassa,
\cite{Boschetti2019}, a quantificação da resiliência de um sistema \cite{Boschetti2019} e análise de
interações entre espécies e a teoria de diversidade funcional \cite{Mouchet2010, Sherwin2019}.

As ferramentas estatísticas utilizadas na área de sustentabilidade não se restringem aos casos
mencionados acima. Um mapeamento detalhado sobre as metodologias usadas atualmente no tratamento de
dados de sustentabilidade é essencial para a compreender a área e vislumbrar novas perspectivas. 

% {\color{red} Escrever algo geral, sobre importância da estatística, análise de dados, predição...
% se for possível OU importância da predição, do uso de modelos estatísticos em sustentabilidade
% (não sei qual a bibliografia mais fácil de encontrar). Se for difícil escrever essa subseção,
% excluímos este item, sem problemas.}

\subsection{Análise Bibliométrica}

%falar sobre filtros e palavras-chave

Conforme discutido em \cite{Ramakrishna2022}, se observa uma crescente produção cientifica na área
de sustentabilidade, de forma que o número de publicações cresce, anualmente, de modo exponencial.
%Com isso, se mostra uma certa tendência que ao vem tomando cada vez mais importância esse assunto e
%mais e mais cientistas estão se dedicando a ele. \par Mais do que nunca vem se observando um
%crescente número de artigos científicos sendo publicados. Com o maior acesso ao ensino superior ao
%redor o mundo, a produção cientifica está mais crescente do que nunca. \cite{Fire2019}. 
Com esse grande número de publicações e a alta velocidade de disseminação do conhecimento, se faz de
extrema importância uma atualização constante, bem como, um olhar mais firme quanto aos conteúdos
divulgados.

Com este grande volume de informações, se tornou cada vez mais difícil fazer uma boa análise
bibliométrica, principalmente manualmente. A medida que essa dificuldade aumentou, cresceu-se também
a necessidade de análises mais amplas e profundas sobre os assuntos que se desejava estudar
\cite{Mongeon2016, Huang2017}. A junção desses fatores, aliado com o desenvolvimento da tecnologia,
fez emergir a análise bibliométrica, capaz de realizar análises científicas mais amplas e profundas
sobre qualquer tópico de interesse. \par

A análise bibliométrica permite mapear e entender o campo de pesquisa de interesse. Para isso, o
pesquisador deve definir alguns critérios de pesquisa, como palavras-chaves, período considerado e
tipos das publicações, dentre outros. A partir desta análise é possível
%estudar relações entre centros de pesquisa, autores e artigos científicos 
encontrar, através de análises estatísticas e matemáticas, as relações, os padrões e as conexões
existentes entre os artigos, autores e centros de pesquisa que publicam sobre o assunto. Além disso,
também é possível analisar
%de relações de autores, onde determinado tópico esta sendo mais
publicado, quais os assuntos de interesse e o impacto de determinado autor ou centro de pesquisa
% teve sobre um determinado assunto.
\cite{Narin1994, Aksu2019}. \par

%Essa analise se tornou possível, em maior escala, com a criação de uma métrica mais unificada para
%a avaliação de trabalhos científicos. Inicialmente, a primeira dessas métricas foi o índice de
%citação cientifica, que hoje em dia, faz parte da Web of Science \cite{Mongeon2016}. Ao longo do
%tempo, aconteceu o desenvolvimento de outras métricas, que podem ser citadas o índice H, o índice
%G, o fator de impacto dentre diversas outras. \par

Portanto, a partir de uma análise bibliométrica, consegue-se identificar os avanços em uma
determinada área do conhecimento, dentro do tópico de interesse, podendo efetuar uma análise mais
profunda sobre as metodologias ou resultados mais importantes, além de se criar um mapa geral sobre
o assunto relevante \cite{Mongeon2016}. \par

Diversas revistas e sites de publicações possuem um banco de dados onde oferecem a possibilidade de
realizar pesquisas considerando critérios definidos pelo pesquisador. Estes filtros incluem campos
como  autoria, área do conhecimento, número de citações, revista publicada, etc. Dentre os sites que
permitem tais buscas pode-se destacar o \textit{Google Scholar}, \textit{Scopus} e \textit{Web of
Science}. Neste projeto, pretende-se realizar a pesquisa da análise bibliométrica a partir do
\textit{Web of Science} e a análise de dados e a construção de gráficos serão feitas usando o
\textit{software R} e o \textit{VoSViewer}.
%Ademais, será necessário fazer toda essa aquisição, sendo indispensável algum tipo de programa que
%consiga faze-lo de forma eficiente. 
\par

\subsection{Ferramentas a serem utilizadas}

\subsubsection{Software RStudio}
O programa que será utilizado para a análise bibliométrica é o \textit{RStudio}. O \textit{RStudio}
é um software livre de ambiente de desenvolvimento integrado para \textit{R}, uma linguagem de
programação voltada para gráficos e cálculos estatísticos, de livre acesso e de fácil utilização
\cite{Rstudio2021}. Este \textit{software} permite realizar uma vasta gama de operações, desde as
mais comuns até operações mais avançadas. \par

O \textit{R} possui uma vasta coleção de pacotes e bibliotecas que podem ser utilizados para
estender suas funcionalidades. Em consonância com o objetivo deste trabalho, destaca-se o
\textit{Bibliometrix}, um pacote poderoso para a análise de dados bibliométricos. \par

\subsubsection{Pacote Bibliometrix}

O pacote \textit{Bibliometrix} é voltado para a análise bibliométrica, sendo amplamente utilizado
por diversos pesquisadores.
%Ela se utilizada de diversas bases de dados para que seja feita a busca analise desses dados,
%dentre esses, incluem o Scopus, Web of Science, Lens.org, etc \cite{Bibliometrix2017}.
Este pacote possui diversas funções pré-programadas que podem ser utilizadas
%dentre elas a busca
por palavras-chaves, autores, revistas, área do conhecimento, etc. Além disso, ele possui funções
que podem ser
% utilizadas para 
permitindo, a partir delas, a criação de gráficos e tabelas de autores, países, centros de estudo e
de artigos, possibilitando a elaboração de tabelas com os artigos ou autores mais influentes na área
no momento da pesquisa. Em junção com o \textit{RStudio}, é possível criar uma análise bibliométrica
completa, com elaboração de gráficos e estatísticas \cite{Bibliometrix2017}. 

Vale salientar que tais resultados devem sempre ser analisados e interpretados pelo pesquisador
responsável pela análise bibliométrica. Uma boa análise bibliométrica apresenta, além destes
resultados, a identificação de \textit{gaps} e tendências emergentes na área de interesse. \par

Junto com o \textit{RStudio / Bibliometrix}, pretende-se também utilizar neste trabalho o pacote
\textit{VOSviewer}.


\subsubsection{Pacote VosViewer}

O \textit{VosViewer} é um software gratuito utilizado para visualização gráfica das relações entre
os artigos, autores, centros de estudos e palavras-chaves. Mais especificamente, o
\textit{VosViewer} é um programa que pode ser utilizado para a visualização de redes de citações, de
co-autoria, de palavras-chaves, etc. Ele consegue gerar mapas de teias de relação, observando-se
onde cada artigo se encaixa e sua relação com outro artigo. Além disso, ele consegue gerar mapas de
clusterização, de calor e de densidade, de forma que se consiga ter uma visão geral sobre o assunto
desejado \cite{VOSviewer2010}. \par

O \textit{VOSviewer} é um programa de fácil utilização e que, com as análises produzidas pelo
\textit{RStudio / Bibliometrix}, fornecem uma análise abrangente sobre a área de pesquisa.

%pode ser utilizado em conjunto com o \textit{RStudio}. , de forma que a busca e analise será feita
%pelo ele e o Bibliometrix, com a visualização sendo feita pelo VOSviewer. \par


\section{Detalhamento das atividades a serem desenvolvidas pelo bolsista}

As atividades a serem desenvolvidas pelo bolsista no decorrer do projeto consistem, na primeira
etapa, de uma análise bibliométrica na área de sustentabilidade, com possível enfoque na(s)
sub-área(s) de resiliência, \textit{flips} ecológicos e/ou quantificação da biodiversidade. Em uma
segunda etapa, o aluno realizará uma leitura detalhada de todos os artigos mais recentes publicados
na(s) sub-área(s) de interesse, identificando as técnicas estatísticas mais utilizadas atualmente.
Por fim, aplicará pelo menos duas das técnicas estatísticas em um conjunto de dados da área de
sustentabilidade. Para a análise de dados, o aluno utilizará o software R.

\section{Resultados previstos e seus respectivos indicadores de avaliação}

O resultado direto esperado com o desenvolvimento das atividades deste projeto é aumentar o
interesse do estudante participante com a pesquisa científica. Além disso, espera-se que ao final do
projeto o estudante tenha uma melhor noção dos conceitos e dos fundamentos específicos dos tópicos
trabalhados, tornando-o ciente das dificuldades e desafios vinculados a problemas de origem
técnico-científicos. Espera-se também que os resultados obtidos neste trabalho possam enriquecer o
entendimento e contribuir na busca do entendimento deste problema extremamente complexo e de caráter
multifatorial.

\section{Cronograma de execução}

\begin{table}[h]
    \begin{adjustbox}{width=1.5\textwidth,center=\textwidth}
    \begin{tabular}{|c||l||c|c|c|c|c|c|c|c|c|c|c|c|}
    \hline
    & & \multicolumn{12}{c|}{\textbf{Mês de execução}} \\ \hline
    \textbf{Etapa} & \textbf{Descrição} & 01 & 02 & 03 & 04 & 05 & 06 & 07 & 08 & 09 & 10 & 11 & 12
    \\ \hline
    1 & \makecell[l]{Definição das palavras-chaves\\ e filtros que serão utilizados\\
    na análise bibliométrica}& \huge{x}& \huge{x}&&&&&&&&&&\\ \hline
    2 &\makecell[l]{Elaboração do código-fonte\\ para análise bibliométrica} & \huge{x}& \huge{x}&
    \huge{x}&&&&&&&&&\\ \hline
    3 &\makecell[l]{Realização da\\ análise bibliométrica}&&&&\huge{x}&&&&&&&&\\ \hline
    4 &\makecell[l]{Organização e interpretação dos\\ resultados obtidos na Etapa
    3}&&&&&\huge{x}&\huge{x}&&&&&&\\ \hline
    5 &\makecell[l]{Leitura completa e detalhada\\ de todos os artigos mais\\
    recentes publicados, identificando\\ as técnicas estatísticas e demais \\
    informações importantes}&&&&&&&\huge{x}& \huge{x}& \huge{x}&&&\\ \hline
    6 &\makecell[l]{Organização dos\\ resultados obtidos na Etapa 5}&&&&&&&&&&\huge{x}&\huge{x}&\\
    \hline
    7 &\makecell[l]{Elaboração do Relatório Final}&&&&&&&&&&&&\huge{x}\\
    \hline
\end{tabular}
\end{adjustbox}
\end{table}
\bibliography{Referencias}

\end{document}


