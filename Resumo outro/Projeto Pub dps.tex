\documentclass{article}[12pt]
\usepackage[utf8]{inputenc}
\usepackage[brazil]{babel}
\usepackage{xcolor,amsmath}


\title{\textbf{Título: O Uso de métricas estatísticas em dados reais}}

\date{}
\author{}
\begin{document}
\maketitle
\section*{Resumo}
(Colocar depois)
\section{Justificativa}
(Colocar Depois)

\section{\textbf{Objetivos}}
(Colocar Depois)

\section{\textbf{Metodologia}}

\subsection{Teoria da Informação}

É consenso que o desenvolvimento e o progresso precisam ocorrer de maneira sustentável, mantendo as condições desejáveis e evitando, assim, mudanças catastróficas na condição do sistema.
É possível que mudanças de regime resultem em danos significativos e, até mesmo, irrecuperáveis (Brock \& Carpenter, 2006). 

Como o objetivo de detectar tais mudanças de regime, surgiu a Teoria da Informação que, dentro de um contexto multidisciplinar, desenvolve e aplica métricas qualitativas e quantitativas para avaliação de mudanças de regime. Tal tópico tornou-se de grande importância para análise de sustentabilidade e, embora não exista um consenso sobre medidas universais, pesquisadores continuam a estudar métricas capazes de detectar e mensurar mudanças de comportamento de sistemas complexos.  Como discutido em (Fisk, 2010), o desenvolvimento sustentável de sistemas humanos envolve a avaliação de uma gama de componentes, incluindo fontes sociais, ambientais e econômicas. Neste cenário, a Teoria da Informação tem se mostrado muito promissora devido à sua capacidade de avaliar a estrutura, complexidade, estabilidade e diversidade do ecossistema (Mayer et al, 2006). 

As metodologias apresentadas a seguir abordam conceitos provenientes da estatística. Em particular, a Informação de Fisher é uma medida de ordem dos dados, sendo capaz de monitorar as variáveis de um sistema e capturar as mudanças de regime (Fath el al, 2003). 


\subsection{Informação de Fisher}

A Informação de Fisher (IF) foi criada pelo estatístico Ronald Fisher, em 1922. Sua motivação inicial foi mensurar a quantidade de
informação do parâmetro $\theta$ de uma distribuição $f$, dado um conjunto de medições,
independentemente, dessa mesma distribuição. 

Note que se $f$ tem um pico acentuado quando varia-se o parâmetro $\theta$, o conjunto de dados $X$ fornece muita informação sobre tal parâmetro. Por outro lado, se $f$ é plana e esparsa, então muitas amostras de $X$ são necessárias para estimar o valor de $\theta$. Isto sugere o estudo de alguma métrica em que se leve em conta a variância do parâmetro $\theta$ (Ly et al,2017).

A Informação de Fisher observada é dada por: %para apenas um parâmetro \(\theta \).\par

\begin{equation}
\label{eq1}
    I = \left(\frac{\partial }{\partial \theta } \ln f(x|\theta ) \right)^{2} 
\end{equation}
onde \(f(x|\theta )\) é a função densidade de probabilidade da variável, dado o parâmetro
\(\theta \). 
%Tal forma inicial tem nome de informação de fisher observada, nos fornecendo, prioritariamente, uma forma de podermos quantificar a quantidade de informação que temos sobre o parâmetro \(\theta \) dado um set de dados retirado de uma distribuição qualquer com parâmetro \(\theta \). 
Valores altos da Informação de Fisher em \label{eq1} são obtidos quando $f$ tem um pico acentuado ao variar-se $\theta$ e indicam que o conjunto de dados fornece grande informação sobre este parâmetro. Já valores baixos no caso em que $f$ é plana e esparsa e indicam que os dados possuem pouca informação sobre o parâmetro de interesse.


Sob outro ângulo, pode-se mensurar a quantidade de informação do parâmetro $\theta$ levando-se em conta a função Score \(S(\theta )\) da distribuição:

\begin{equation}
\label{eq2}
S(\theta)=\frac{\partial}{\partial \theta} \ln (f(x| \theta )^{2})
\end{equation}


Da mesma forma que na Equação (1), esta função, em suma, nos diz o quanto distribuição $f$ é sensível a uma mudança pequena sobre o parâmetro \(\theta \). 

{\color{red} Parei aqui! Novos comentários...}

A idéia de R. Fisher para quantificar nossa informação foi pegar o
valor médio da curvatura da nossa função score, que também pode ser pensado como a variância da
nossa função score (Pawitan, 2006). Tal análise nos gera a informação de fisher esperada, denotada por \(\mathcal{I} \). \par

\begin{equation}
    \mathcal{I} = E\left[\frac{\partial}{\partial \theta } \ln (f(x| \theta)^{2})\right]
\end{equation}

Para relacionar os dois tipos de informação de fisher, basta observar que \(\mathcal{I} \) é o valor
médio de \(I\). Logo, se fossemos explicar, \(I\) seria uma função, que nos diria a curvatura da
função score em um determinado \(\theta \) e consequentemente sua informação e \(\mathcal{I} \) nos
diria essa curvatura média, e consequentemente, sua informação média (Pawitan, 2006). Ademais, por consequência da
nossa definição, \(\mathcal{I} \) é independente da quantidade de dados que coletamos, podendo ser
usado em várias coletas diferentes. \(I\) por sua vez é bastante dependente do conjunto de dados,
variando de coleta em coleta. \par

\begin{equation}
    \mathcal{I} =E(I)
\end{equation}

Ademais, por meio da desigualdade de Cramer-Rao, a informação de
fisher se relaciona com o erro que cometemos na nossa estimativa \(\hat{\theta} \) por meio de 
\begin{equation}
    e\mathcal{I} \geq 1
\end{equation}
Ou seja, se tivermos uma informação de fisher muito grande, necessariamente teremos que ter um erro
muito baixo. Então na estatística de inferência se é muito interessante tentar maximizar nossa
informação de Fisher para um determinado parâmetro como meio de diminuir nosso erro cometido (Pawitan, 2006).
Ademais, algumas considerações é que o valor da informação de fisher é sempre positiva e maior ou
igual a zero e não é afetada por uma mudança de coordenadas, no sentido de mudarmos para direita ou
esquerda nossa distribuição (Frieden, 2010).

Quando vamos tratar de sistemas dinâmicos, sua grande maioria é dependente do tempo, sendo então
necessário uma forma de inclusão na nossa métrica  (Gonzalez-Mejia et al., 2015). Incluindo um termo para o tempo, nossa equação
fica:

\begin{equation}
    \mathcal{I}  = \int_{-\infty}^{\infty} f(x|\theta ) (\frac{\partial }{\partial \theta } \ln f(x|\theta ))^{2}(\frac{\mathrm{d}x}{\mathrm{d}t})^{-1}   \,\mathrm{d}x
\end{equation}

É muito difícil ter as funções necessárias para a informação de fisher, onde para muitos datasets
reais, temos informações faltando, com um certo grau de imprecisão, dentre outros problemas (Gonzalez-Mejia et al., 2015). Se
desenvolveu então métodos para ser calculado essa informação para que seja aplicável em situações
reais. O primeiro é a utilização da forma continua da equação, estimando-se os valores necessários e
O segundo é por meio de um 'empacotamento' dos dados.


\subsection{Formas de cálculo de FI}

Para o método contínuo, utilizando algumas condições de regularização podemos definir qualquer
sistema como seguindo uma certa trajetória em um espaço n-dimensional, ligado à quantidade de
variáveis. Utilizando-se um pouco de física, esse método manipula nossa equação original para
utilizar da velocidade e aceleração tangencial do nosso sistema, utilizando seu cálculo como meio de
avaliar a informação de fisher (Mayer et al., 2006). \par

\begin{equation}\label{eq: forma vel acel}
    \int_{-\infty}^{\infty} \frac{(\mathrm{d}x/ \mathrm{d} x)^{2} }{(\mathrm{d}^{2} x/\mathrm{d}x^{2})^4 } \,\mathrm{d}x 
\end{equation}

Como nossos pontos de dados são discretos, precisaremos estimar e aproximar nossos termos de
derivadas de primeira e segunda ordem. Para isso, podemos usar métodos como a série de Taylor, onde
tenta-se encaixar a maior quantia de dados possíveis (Gonzalez-Mejia et al., 2015). Mas claro que ainda está suscetível a erros,
principalmente a segunda derivada, a qual é muito sensível à barulho nos dados e pode causar
diversos problemas caso isso não seja remediado.\par

Existem algumas formas mais comuns que são usadas nisso, dentre elas está tentar achar uma função de
melhor encaixe para os dados e a partir dela usar métodos analíticos. Mas claro que esse método
também sofre de problemas, que talvez nossos dados sofram um comportamento que uma curva não consiga
aproximar tão bem e problemas de generalização, onde se quisermos estender nossas análise, para
talvez prever o que pode acontecer, a generalização da nossa curva pode não ser boa (Gonzalez-Mejia et al., 2015). \par

Outra forma é fazermos o calculo para um período \(T\) suficientemente grande de tal forma que esses
barulhos não interfiram na nossa estimativa (Mayer et al., 2006). Mas claro que, sofre de problemas, onde perdemos
detalhes dos nossos dados. Pode não ser problemático, mas caso seja uma mudança muito pequena que
influencie, a análise não seria útil. \par

Uma das grandes necessidades desse método é que se precisa calcular a informação de fisher para no
mínimo um ciclo do sistema, ou seja, precisaria estimar o quanto um ciclo representaria na escala do
tempo. Porém, para sistemas complexos, como os reais, isso é muito difícil, portanto foi
desenvolvido uma estimativa que se aproxima muito do ciclo real, já que o valor exato não é possível
ser calculado, com certas limitações (Rawlings et al., 2020). A não utilização de um período exato, ou próximo, nos
gera um gráfico de fisher com muita oscilação e com uma interpretação muito difícil de ser
feita.\par

Dado essas problemáticas, também existe o método discreto do cálculo da nossa Informação da Fisher  (Karunanithi
et al., 2008).
Porém, invés de usarmos nossa fórmula já descrita, usamos a forma de amplitude dela, de forma que
fazendo uma substituição de funções, \(p(s)=\sqrt{g(s)} \), nossa equação fica:
\begin{equation}
    I=4\int_{-\infty}^{\infty} \frac{\mathrm{d}q(s)}{\mathrm{d}t} ^{2}  \,\mathrm{d}s 
\end{equation}
Onde faremos uma substituição por quantidades discretas, onde nossa integral se torna um somatório e
nossas quantias \(\mathrm{d}s, \mathrm{d}t \) se tornam quantias reais e finitas, que dada certas
condições, nossa aproximação fica
\begin{equation}
    I \approx 4\sum_{i}^n [q_i - q_{i+1} ]^2
\end{equation} 

A estratégia desse método é tentar reduzir as variáveis de estados agrupando-as quando obedecem uma
certa regularização. Para cada variável medida, existe um erro em sua medição. Se a diferença entre
a mesma variável, em diferentes tempos, for menor que o erro para essa variável, consideramos que
São diferentes observações para o mesmo valor e o agrupamos. Agora, se o mesmo valor observado em
diferentes variáveis foi observado, ou seja, se diferenciam por menos que a incerteza, agrupando-as em
um grupo (Karunanithi
et al., 2008). Cada variável representaria um ponto num hiperplano que contenha todas as variáveis.
Portanto, para cada variável agrupada, faremos um volume de tal forma que a probabilidade de
observamos um estado seria proporcional à quantidade de pontos que esse volume estaria contendo. Dessa
forma, conseguimos estimar cada probabilidade de um estado ser observado e calcular a informação
de fisher. \par

Porém, esse método tem o problema de necessitarmos dessa incerteza para cada variável. Uma forma de
fazer isso é observar um sistema que já esteja estável e contenha as mesmas variáveis e usarmos a
variação dessas variáveis como incerteza para nossos cálculos. Caso isso não seja possível, teremos
que achar o estado mais estável do nosso sistema em estudo e calcular o desvio padrão para cada
variável e assumi-lo como incerteza. (Gonzalez-Mejia et al., 2015,Karunanithi
et al., 2008 ) \par

Posteriormente, se calcula a evolução da função densidade de probabilidade com o tempo, dividindo o
período em janelas de tempo e calcula-se a probabilidade de cada estado e a forma discreta de
amplitude da nossa informação de fisher. Tal método também sofre do problema de necessitar do
período, mas é bastante útil no sentido de conseguir lidar com variações e incertezas nos nossos
dados.

\par


\subsection{Aplicações no Real}

O uso da informação de Fisher vem sendo estudado a um certo tempo com resultados bastantes promissores. No artigo seminal publicado em 2006 por Audrey L. Mayer, et al. Eles propõem uma forma de uso e aplicações da Informação de Fisher.
Por meio o uso da formula \eqref{eq: forma vel acel}, fazendo-se estimativas numéricas para a velocidade e aceleração, conseguiram com sucesso o calculo para diversos ecossistemas a qual se tiveram mudança de regime. \par

A ideia da analise para essa informação é analisar a variação da Informação de Fisher ao longo do tempo, de forma que se o sistema está em um ciclo estavel, seu valor não se alteraria ao longo do tempo. Porem, caso se tenha uma variação, ou seja, o sistema saísse do seu ciclo, classificaríamos como um flip ecologico.
Foram feitas algumas analises em sistemas que por meio de outros métodos foi-se identificado essas mudannças ecologicas. Portanto, a Informação de Fisher, precisaria ter uma mudança, positiva ou negativa, na faixa dessas mudanças. Para o ecossistema do norte do pacifico, na região proxima do estreito de bering, se tiveram duas mudanças ecologicas, uma entre 1976 e 1977 e outra em 1989.
Utilizando-se uma base de dados com 100 variaveis, com a seleção de cerca de 70, foram-se feitas ás analises e estimativas e observou-se uma clara mudança do valor da Informação de Fisher coicidindo com os flips ecologicos ja estabelecidos por outras metodologias.\par

Ademais, foi feita a analise sobre o clima global, onde analisou-se niveis de gases, como o dioxido de carbono e metano, que ficaram presos no gelo da antartica ao longo dos milhares de anos. A quantia deses gases se relacionam diretamente com o clima mundial daquela epoca, de forma que servem como um bom indicador de mudancas climaticas em escala global. Mudancas globais sao vistas especialmente entre climas majoritariamente frios e majoritariamente quentes, portanto, foi evidenciado umamudanca da Informacao de fisher entre esses periodos. \par

Por fim, a ultima analise que foi feita foram nos climas Saharianos, onde ambos tiveram umamudanca similar entre climas aridos e umidos. Por meio de coleta de sedimentos de lagos e oceanos, foram coletados dados como porcentagem de poeira terrestre no leito oceanico {\color{cyan} Pesquisamr mais variaveis que foram analisadas} e por meio delas foi feito o calculo da informacao. As mudancas observadas se relacionam as mudancas entre regimes e estao de acordo com estimativas de mudancas nesse sistema por meio de outras metodologias.\par

Apesar do aparente sucesso dessa metrica, foram ressaltados algumas limitacoes. Por mais que se tenha a mudança, nao foi muito bem definido se a variacao foi negativa ou positiva seria de significancia para a deteminacao do regime. Ademais, a dificuldade da estimativa do periodo foi um fator limitante quanto o calculo sendo que ja era sabido previamente um valor para esse periodo, que caso queira analisar para novos ecossistemas, esse valor nao seria conhecido.
{\color{cyan} Parei aqui} \\

Entropia de Shannon:
(Pesquisar e colocar depois)
Outra Métrica:
(Pesquisar outras possíveis métricas)
\end{document}