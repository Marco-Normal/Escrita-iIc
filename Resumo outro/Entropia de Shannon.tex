\chapter{Entropia de Shannon}
\section{Uma Introdução ao conceito de entropia}
\subsection{A diferença entre Termodinamcia e Estatistica}
Bom, inicialmente muito se fala quanto a entropia, porém muito no sentido termodinamico. Em suma é
uma medida da quantidade de estados que as moleculas de uma determinada podem estar. No exemplo
classico, a agua no estado solido é aquela que apresenta menor entropia, a liquida com uma entropia
média e o vapor com uma entropia alta. Tal valor, geralmente denominado de \(S\) é calculado pela
formula de Boltzmann para gas ideal, mas muitas vezes pode ser calculada e observada
empiricamente.\par

Enquanto a entropia de Shannon, num sentido mais estatistico e de teoria da informação, tem uma
certa relação com isso, não será o foco da escrita. Mas claro, não se poderia deixar passar batido
essa comparação. Em seu trabalho seminal, lançando em 1948, onde ele coloca as fundações da teoria
da informação, tal trabalho de Shannon foi muito produto de sua epoca, pois a grande maioria do que
foi colocado foi desenvolvido para ser usado na segunda guerra mundial, tanto para classificar a
informação de transmissão de radios, quanto para transmitir mensagens codificadas, quanto para
decifra-las. Enquanto tambem não o foco do que sera escrito, é interessante saber o contexto
historico \par.

\subsection{Informações e Bits}
Em computação vemos os bits em tudo quanto é lugar. Nossa velocidade da internet é medida em bits,
em geral uma ordem de grande superior, megas ou até gigas, e nosso armazenamento tambem, geralmente
em gigas ou teras. Mas o que realmente são bits? Imaginemos que, por exemplo, uma variavel possa
assumir apenas dois valores 0 ou 1. Ela teria apenas um bit de informação, pois assumiria apenas 2
valores. Se tivermos apenas uma letra, por exemplo, quantos bits ela teria? Se tivermos \(n\) tipos
de informações, vamos precisar de \(\log_{2}(n) \) bits. Para um alfabeto de \(26\) letras, vamos
precisar de \(\log_2(26) \approx 4.7\) para essa uma letra. Isso supondo que cada letra tenha uma
probabilidade igual de ser escolhida. \par

Vamos pensar agora como podemos construir para nossa entropia de Shannon. Com mais um exemplo de
bits. Com a mesma ideia do alfabeto, se quisermos, por exemplo, escrever uma frase com 5 palavras,
quantos bits precisariamos? Para cada letra, teriamos 4.7 bits, se cada uma delas for independente e
de igual probabilidades, podemos so pegar e somar 4.7 5 vezes. De fato, podemos fazer isso para
qualquer palavra, desde que não tenha espaço, ou caracteres especiais, claro. Então para uma palavra
de \(m\) letras, precisariamos de \(m*4.7\) bits. Vamos pensar então, é muito mais facil, por
exemplo, tentar acertar uma palavra de 3 letras, do que uma palavra de 10. A quantidade de bits está
relacionado com essa 'facilidade'? Mantenhamos esse questionamento em mente

\subsection{Conhecimento}

Continuando nossa analogia no alfabeto. Vamos supor que temos uma palavra de 5 letras e você precise
decifra-la. Inicialmente você não tem conhecimento nenhum, todas as letras são igualmente provaveis.
Mas agora, suponha que eu te diga, existe a letra a nessa palavra. Você ja ganhou algum
conhecimento. Mas você conseguiria quantifica-lo? Em uma ideia mais trivial e popular, você pode até
pensar, tal letra é bastante comum na lingua portuguesa, foi-se revelado, obviamente, que apenas
palavras contendo apenas as outras vogais são validas. Não reduziu-se tanto, então, nosso espaço de
possibilidades. Por outro lado, se eu te falasse que contem a letra z. Já é uma boa ajuda, pois a
letra z é bastante incomum, então você acaba ganhando mais informação no geral quando te falo tal
dado. Para quantificar esse conhecimento ganho, podemos colocar como
\begin{equation}\label{eq:Ganho de informacao}
    G(B|A)=\log _2 \frac{1}{p(B)}=-\log _2(p(B))
\end{equation}
Onde temos A e B como um espaço de probabilidade, tal que \(B \subset A\). O ganho de informação é
sempre positivo. Ademais, ele obdece a propriedade que se \(B\subset C\subset A\) , temos que
\begin{align}
    G(C|A)=-\log _2p(C)\\
    G(B|C)=-\log _2p(B|C)=-\log _2\frac{p(B)}{p(C)}
\end{align}
Ou seja, o ganho que se tem descobrir que ele pertence a \(B\) depois de saber que pertence a \(C\)
é a razão das probabilidades de \(B\) e \(C\). Por consequencia, \(G(B|A)=G(C|A)+G(B|C)\) Portanto,
dado o espaço de probabilidades \(A=(a_1,a_2,...,a_n)\), se tivermos um ganho dado por
\(G=-\log p(a_n)\) o ganho médio esperado é dado por:
\begin{equation}\label{eq:Entropia de shannon}
    H(A)=\sum_i p(a_i) -\log p(a_i)
\end{equation}
Esse ganho médio tambem pode ser entendido como a quantidade de informação da função densidade de
probabilidade. Ela não esta relacionada, exclusivamente, com a variavel aleatoria, mas sim com sua
distribuição. Não é incomum ve-la escrita como \(H(p)\) inves da forma que foi escrita, para não ser
criada uma confusão.


\subsection{Diferenças entre Informação de fisher e Entropia de Shannon}

Lembramos que a informação de fisher se relaciona à curvatura média da nossa função score Como isso
se relaciona com a entropia. Se fossemos pensar em um sentido mais empirico quanto maior a nossa
informção, menor seria o nosso caos, ou seja menor seria nossa entropia. Então necessariamente a
informaçoinformação de fisher seria o inverso da entropia de shannon Não necessarimente. Existe uma
diferença fundamental entre os dois. A entropia de shannon fundamentalmente se relaciona com o
conteudo da mennsagem, quanto de informação ela carrega. Porem, por exemplo, digamos que você receba
uma mensagem Completamente aleatoria, por assim dizer, euma mensagem com um texto claro. Qual delas
teria a menor entropia: Claramente é aquela que vem com a menagem claramente mais identificada, já
de antemão para nos. Porem, se fossemos decifrar as duas mensagens, qual você aprenderia mais: Seria
aquela de maior entropia pois se conseguissemos identificar o que faz com que ela fique aleatoria e
conseguissemos reproduzi-la de forma similar, aprenderiamos muito mais. Porém geralmente se assume
que transmissões com maior entropia carregam menos infromações pois demandam menos bits para serem
transportados. Dentro da teoria original de fisher ele não expressou uma relação direta com a
entropia, porem, posteriormente foi-se expressado tal relação. Segue que:

\subsection{Pontos importante na Entropia de Shannon}
Até o momento so fizemos a analise de um fator, ou seja \(H(A)\), onde apenas consideramos a
distribuição de probabilidades \(A\), mas se por exemplo, fomos considerar um segundo fator, ou
seja, uma distribuição, \(B\), como ela interagiria com \(A\). Chamando \(p(a_i)\) como a
probabilidade de \(a_i\) acontecer e \(p(b_j)\) como a probabilidade de \(b_j\) acontecer, temos
alguns tipos de probabilidades conjuntas. Por exemplo, \(p(a_i,b_j)\) sendo a probabilidade de
\(a_i,b_j\) acontecerem simultaneamente, \(p(a_i|b_j)\) sendo a probabilidade condicional de \(a_i\)
acontecer dado que \(b_j\) aconteceu. Chamando a informação media conjunta de \(M(A,B)\), pode
ser mostrado que:
\begin{equation}
    M(A,B)=\sum_{i,j} p(a_i,b_j)\log \frac{p(a_i|b_j)}{p(b_j)}
\end{equation}
De forma que pode ser mostrado que \(H(A)\geq M(A,B)\geq 0\) . A interpretação de \(M(A,B)\) é que
ele nos diz quanto da complexidade de \(H(A)\) é explicada por \(B\) , ou seja, o quanto de \(A\) é
restringido por \(B\). Ademais, podemos decompor a entropia de Shannon como sendo
\begin{equation}\label{eq:entropia de shannon decomposta}
    H(A)=M(A,B)+H(A|B)
\end{equation}
Onde \(M(A,B)\) mostra o quanto \(A\) é restringido por \(B\) e \(H(A|B)\) fala a 'liberdade' de
\(A\) dado a presença de \(B\) . Em Um sistema ecologico, então, isso nos mostraria, por exemplo,
num sistema presa predator, onde ele é muito restringido pela eficiencia de caça e comida, a
entropia de Shannon exprime tanto a eficiencia quanto a liberdade. Liberdade essa que se
relacionaria à liberdade do predador dado a presa.