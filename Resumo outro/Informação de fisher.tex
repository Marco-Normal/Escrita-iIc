\chapter{Uma analise profunda da informação de Fisher}
\section{Um passo para trás}
Eu acho que é importante para realmente entender e fazer as corretas derivações necessarias, ter o
conhecimento de infomações que precedem o nosso objetivo final, tanto para realmente entender, não
so copiar uma formula, tanto para entender de onde possivelmente são os erros e possiveis locais de
melhoria. Então por enquanto, farei uma pesquisa sobre as materias que levam à informação de fisher.
\subsection{Estatistica suficiente}
Colocando que temos um conjunto de dados \(X_1,X_2,\ldots ,X_n\) gerados por uma função de
probabilidade qualquer com parametro \(\theta \), se existir uma estatistica
\(Y=u(X_1,X_2,\ldots,X_n)\), tal que a condicional \(P(X_1,X_2,\ldots ,X_n|Y=y)\) não depender de
\(\theta \). Exemplo: \par
Dado uma distribuição de bernoulli, com parametro \(\theta \) desconhecido, onde se fazem n testes
tal que existam n dados. Uma estatistica suficiente para esse parametro é a soma de todos os
\(X_i\), pois:
\begin{align*}
    Y=\sum_{i} X_i \\
    P(X_1,X_2,\ldots ,X_n|Y=y)&=\frac{p(X_1=x_1,X_2=x_2,\ldots ,X_n=x_n,Y=y)}{P(Y=y)}\\
    &=\frac{P(X_1=x_1,X_2=x_2,\ldots ,X_n=x_n,\sum_{i} X_i=x_i)}{P(\sum_{i} X_i=x_i)}
\end{align*}
Não dependendo do parametro \(\theta\). Ou seja, conseguimos toda informação possivel sobre o
parametro, apenas com essa estatistica, pois independente dele, adicionando, retirando dados,
nenhuma informação a mais sobre o parametro será ganhada. \par

Infelizmente, essa condição não nos da o necessario para calcularmos que estatistica será essa,
então ainda é um grande de um trabalho imaginarmos isso. Por sorte, temos um teorema que nos ajudará
nesse quesito, o \textit{Teorema da Fatorização} em que ele fala que se uma estatistica
\(Y=u(X_1,X_2,\ldots ,X_n)\), so é suficiente se e apenas se, a distribuição
\(f(x_1,x_2,\ldots,x_n;\theta )\), que gerou dos dados, pode ser fatorizada em dois fatores,
\(\phi,h \), da forma \(f(x_1,x_2,\ldots,x_n;\theta )=\phi
(u(x_1,x_2,\ldots,x_n);\theta)h(x_1,x_2,\ldots,x_n)\), onde \(\phi\) que é dependente dos dados
apenas por meio de \(u(x_1,x_2,\ldots,x_n)\) e \(h\) é independente do parametro. Vamos à um
exemplo:
Suponha uma bernoulli de parametro \(\theta \) desconhecido e constante. Coletando n dados de forma
independente, podemos achar nossa estatistica de modo
\begin{align*}
    f(x_1,x_2,\ldots ,x_n;\theta)&=f(x_1,\theta)*f(x_2,\theta)*\ldots*f(x_n,\theta)\\
    &=\theta^{x_1}(1-\theta)^{1-x_1}*\theta^{x_2}(1-\theta)^{1-x_2}*\ldots*\theta^{x_n}(1-\theta)^{1-x_n}\\
    &=\theta^{\sum_{i} x_i}(1-\theta)^{n-\sum_{i}x_i} 
\end{align*}
O parametro \(\theta \) apenas interage com a estatistica e a função \(h\) é uma constante de valor 1.
\subsection{Maxima Verossemelhança}
Outro tópica que sera discutido e posteriormente usado será o de maxima verossemelhança. Esse
conceito é muito usado em questão de estimação de parametros e estatistica Bayesiana e nos servira
bem para o proposito final. \par

Supondo que temos uma certa quantia de dados, vindos de uma distribuição qualquer, com um parametro
\(\theta \), fixo, com valores podendo estar num intervalo fechado ou não. Idealmente, gostariamos de
tentar achar a curva de melhor encaixe nesses dados. Por facilidade, vamos colocar que sabemos que
eles se originam de uma curva normal de parametros desconhecidos. Sabemos a priori que a curva
normal está centrada na média e sua dispersão e peso da cauda relaciado à sua variacia. Seus
parametros pode assumir qualquer valor na reta dos reais, então, em tese, teriamos infinitas curvas
que se encaixariam nos nossos dados. Porem, podemos supor, até mesmo empiricamente, que alguma delas
tem que se encaixar a melhor, até por que nossos dados foram tirados dessa curva com parametros
fixos. Como poderiamos tentar determinar, ou testar, se algum parametro é melhor do que outro? \par

Para isso, temos a ideia das chances e da maxima verossemelhança. Vamos começar com as chances.
Elas, diferente do entendimento popular, são diferentes das probabilidades. Elas não seguem as
mesmas regras, de que, por exemplo, tem de ser somada 1 no seu inteiro. Alem disso, elas se diferem
de uma probabilidade condicional por uma constante K, da forma L(\(\theta\))=K*P(\(X=x_i|\theta \)).
Onde \(\theta \) seja nosso parametro, ou parametros. Sozinha, não temos muita interpretação passa
essa chance, dado a constante K que não tem valor de interpretação para nós. Então, ela vem sempre a
titulo de comparação para nos. Nossa chance, é em relação ao parametro, então é justo assumir que
uma comparação entre duas chances, na forma de razão, nos diria o quanto uma é superior a outra.
Supondo por exemplo, que a comparação L(\(\theta_1\))/L(\(\theta_2\) ) nos de um valor de 1.5.
Significaria que os dados tem 1.5 vezes maior probabilidade de estarem sobre o parametro
\(\theta_1\) do que do parametro \(\theta _2\).\par

Isso nos leva a pensar, existiria algum parametro tal que sua razão de chances sempre dê um valor
maior que 1? Sim, isso é dito pela Lei da verossimilhança. \textit{Dentro de um modelo estatistico,
um conjunto de dados sempre vai se encaixar melhor em um parametro do que em outro, tal que as
chances do primeiro parametro sempre vai ser maior que a do segundo.}\par
\subsection{Chances}
Enquanto ja é um começo, isso não nos da uma função clara de como achar esse parametro. Para isso,
temos alguns meios de seguirmos. O primeiro, usando um pouco de calculo, é achar o maximo e minimo
de funções. Como temos uma função das chances pelo valor de \(\theta \) podemos tentar achar o nosso
pico da função. Lembrando de calculo, seria derivar e igualar a zero. Porem, é muitas vezes mais
facil achar o maximo do log da função. Maximizar o log da função é igual a maximizar a função. Para
uma binomial, por exemplo, se quisermos estimar o parametro \(\theta \) fazemos
\begin{gather*}
    P(X=x_i)=\binom{x}{n}\theta^x*(1-\theta)^{n-x}\\
    \log (P(X=x_i)) = \log (\binom{x_i}{n}) +x\log (\hat{\theta} )+(n-x)\log (1-\hat{\theta} )\\
    \frac{\mathrm{d}P}{\mathrm{d}\hat{\theta} } =\frac{\mathrm{d}}{\mathrm{d}\hat{\theta} } \log (\binom{x_i}{n})+
    \frac{\mathrm{d}}{\mathrm{d}\hat{\theta} }x\log (\hat{\theta} )+\frac{\mathrm{d}}{\mathrm{d}\hat{\theta} }(n-x_i)\log (1-\hat{\theta})\\
    0=\frac{x_i}{\hat{\theta} }-\frac{n-x_i}{1-\hat{\theta} }\\
    n\hat{\theta}-x\hat{\theta}=x-x\hat{\theta}\\
    \hat{\theta}=\frac{x}{n}
\end{gather*}
Se nossa distribuição fosse multimodal, uma solução explicitamente analitica talvez não fosse
possivel, então metodos numericos teriam de ser utilizados.\par
\subsection{Desigualdade de Cramer-Rao}
Se tivermos um conjunto de dados todos retirados de uma mesma função de probabilidade na forma
\(f(\vec{x},\theta )\), onde se tem n dados e 1 parametro. Se cada dado for independente, pela regra
de probabilidade, sua distribuição conjunta é \(f_{\mathbf{X}}(\vec{x},\theta)\). Similarmente, seu
produto de chances é dado por \(L_{\mathbf{X}}(\theta )=\prod_{i=1}^{n}= f_{\mathbf{X}}(x_i,\theta )\). Supondo
agora, que \(L_{\mathbf{X}}(\theta)\) seja diferenciável em todo espaço e não dependa de \(\theta
\), mais especificadamente, não tenha suporte em \(\theta \). Se integramos a função de chances,
\(L_{\mathbf{X}} \) em todo espaço, teremos o valor de 1. Isso é dado que ela é so o produtorio de
todas as funções de densidade. O produto de varias funções de densidade nos da uma função de
densidade e por definição, em todo espaço, sua probabilidade é 1. Ou seja
\begin{align}
    \int_{-\infty}^{\infty} \int_{-\infty}^{\infty} \ldots \int_{-\infty}^{\infty}  \,  L_{\mathbf{X}}(\theta ) \,\mathrm{d}x_1  \,\mathrm{d}x_2 \,...\mathrm{d}x_n&=1\\
    \intertext{Derivando-a, parcialmente em \(\theta\) temos:}\notag\\
    \frac{\partial}{\partial \theta } \int_{-\infty}^{\infty} \int_{-\infty}^{\infty} \ldots \int_{-\infty}^{\infty}  \,  L_{\mathbf{X}}(\theta ) \,\mathrm{d}x_1  \,\mathrm{d}x_2 \,...\mathrm{d}x_n&=0\\
    \intertext{Como definimos que não há suporte em \(\theta \) nas nossas distribuições, podemos colocar a
    diferencial dentro da integral.}\notag\\  
    \int_{-\infty}^{\infty} \int_{-\infty}^{\infty} \ldots \int_{-\infty}^{\infty}  \, \frac{\partial}{\partial \theta } L_{\mathbf{X}}(\theta ) \,\mathrm{d}x_1  \,\mathrm{d}x_2 \,...\mathrm{d}x_n&=\\
    \intertext{fazendo uma certa manipulação albegrica, temos}\notag
    \int_{-\infty}^{\infty} \int_{-\infty}^{\infty} \ldots \int_{-\infty}^{\infty}  \,\frac{1}{L_{\mathbf{X}}(\theta )} \frac{\partial}{\partial \theta } L_{\mathbf{X}}(\theta ) L_{\mathbf{X}}(\theta )\,\mathrm{d}x_1  \,\mathrm{d}x_2 \,...\mathrm{d}x_n&=\\
    \int_{-\infty}^{\infty} \int_{-\infty}^{\infty} \ldots \int_{-\infty}^{\infty}  \,\frac{\partial}{\partial \theta } \ln (L_{\mathbf{X}}(\theta )) \cdot L_{\mathbf{X}}(\theta )\,\mathrm{d}x_1  \,\mathrm{d}x_2 \,...\mathrm{d}x_n&=  (\theta )\\
    \theta \int_{-\infty}^{\infty} \int_{-\infty}^{\infty} \ldots \int_{-\infty}^{\infty}  \,\frac{\partial}{\partial \theta } \ln (L_{\mathbf{X}}(\theta )) \cdot L_{\mathbf{X}}(\theta )\,\mathrm{d}x_1  \,\mathrm{d}x_2 \,...\mathrm{d}x_n&=\\
    \label{eq:1 da fisher}
    \int_{-\infty}^{\infty} \int_{-\infty}^{\infty} \ldots \int_{-\infty}^{\infty}  \,\frac{\partial}{\partial \theta } \ln (L_{\mathbf{X}}(\theta )) \cdot L_{\mathbf{X}}(\theta ) \cdot \theta \,\mathrm{d}x_1  \,\mathrm{d}x_2 \,...\mathrm{d}x_n&=
\end{align}
Guardemos essa nossa equação. Vamos trabalhar agora estimando nosso parametro \(\theta \). Seja
\(\hat{\theta} \) um estimar do meu parametro orginal, tal que \(E(\hat{\theta}) =\theta  \). Por
definição de esperança, podemos reescrever a equação como:
\begin{align}
    \int_{-\infty}^{\infty} \int_{-\infty}^{\infty} \ldots \int_{-\infty}^{\infty}  \, \hat{\theta}  L_{\mathbf{X}}(\theta ) \,\mathrm{d}x_1  \,\mathrm{d}x_2 \,...\mathrm{d}x_n&=\theta\\
    \frac{\partial }{\partial \theta } \int_{-\infty}^{\infty} \int_{-\infty}^{\infty} \ldots \int_{-\infty}^{\infty}  \, \hat{\theta}  L_{\mathbf{X}}(\theta ) \,\mathrm{d}x_1  \,\mathrm{d}x_2 \,...\mathrm{d}x_n&=\frac{\partial }{\partial \theta } \theta\\
    \int_{-\infty}^{\infty} \int_{-\infty}^{\infty} \ldots \int_{-\infty}^{\infty}  \, \hat{\theta}   \frac{\partial }{\partial \theta }L_{\mathbf{X}}(\theta ) \,\mathrm{d}x_1  \,\mathrm{d}x_2 \,...\mathrm{d}x_n&=1\\
    \label{eq:2 da fisher}
    \int_{-\infty}^{\infty} \int_{-\infty}^{\infty} \ldots \int_{-\infty}^{\infty}  \, \hat{\theta}   \frac{\partial }{\partial \theta }\log (L_{\mathbf{X}}(\theta) ) \,\mathrm{d}x_1  \,\mathrm{d}x_2 \,...\mathrm{d}x_n&=
\end{align}
Subtraindo as equações \eqref{eq:2 da fisher} e \eqref{eq:1 da fisher}
\begin{align}
    \int_{-\infty}^{\infty} \int_{-\infty}^{\infty} \ldots \int_{-\infty}^{\infty}  \, (\hat{\theta}-\theta) \frac{\partial }{\partial \theta }\log (L_{\mathbf{X} }(\theta) ) \,\mathrm{d}x_1  \,\mathrm{d}x_2 \,\ldots ,\mathrm{d}x_n&=1\\
    \int_{-\infty}^{\infty} \int_{-\infty}^{\infty} \ldots \int_{-\infty}^{\infty}  \, (\hat{\theta}-\theta) \frac{\partial }{\partial \theta }\log (L_{\mathbf{X} }(\theta) ) \,\mathrm{d}x_1  \,\mathrm{d}x_2 \,\ldots, \mathrm{d}x_n&=\\
    (\int_{-\infty}^{\infty} \int_{-\infty}^{\infty} \ldots \int_{-\infty}^{\infty}  \,\sqrt{L_{\mathbf{X} }(\theta)} (\hat{\theta}-\theta) \frac{\partial }{\partial \theta }\log (L_{\mathbf{X} }(\theta))\sqrt{L_{\mathbf{X} }(\theta)} \,\mathrm{d}x_1  \,\mathrm{d}x_2 \,\ldots, \mathrm{d}x_n)^{2}&=
    \intertext{Lembrando de algebra linear, tal forma é reconhecida como um produto escalar entre dois vetores
    onde podemos, então, separa-los e aplicar a regra de cauchy-schwartz}\\ \notag
    \int_{-\infty}^{\infty} \int_{-\infty}^{\infty} \ldots \int_{-\infty}^{\infty}  \,L_{\mathbf{X} }(\theta) (\hat{\theta}-\theta)^{2} \,\mathrm{d}x_1  \,\mathrm{d}x_2 \,\ldots, \mathrm{d}x_n& \cdot \\
    \int_{-\infty}^{\infty} \int_{-\infty}^{\infty} \ldots \int_{-\infty}^{\infty} (\frac{\partial }{\partial \theta }\log (L_{\mathbf{X} }(\theta)))^{2} L_{\mathbf{X} }(\theta) \,\mathrm{d}x_1  \,\mathrm{d}x_2 \,\ldots, \mathrm{d}x_n& \geq 1 \notag
\end{align}
A primeira parte reconhecemos como a variancia dos nossos dados. A segunda parte, nada mais é
que a definição da informação de fisher, onde é o valor esperado, esperança, da função score
\begin{equation}
    \label{eq:desigualdade de Cramer-Rao}
    Var(\theta ) \cdot E[\frac{\partial}{\partial \theta } \ln (L(\theta )^{2})L(\theta)]\geq 1 
\end{equation}
Alem de mostrar de onde surge a equação de fisher, foi mostrado tambem outro ponto muito importante,
que foi a desigualdade de Cramer-Rao. Essa desigualdade mostra que a informação de fisher é
inversamente proporcional ao erro, ou variancia do nosso modelo. Ou seja, com uma informação de
fisher extremamente alta, nosso erro sera extremamente baixo e vice-versa \par

Falando um pouco mais sobre a função score. Ela é definida da seguinte forma:
\begin{equation}
    S(\theta)=\frac{\partial}{\partial \theta } \ln L(\theta )
\end{equation}
Seu valor tende a 0 quando nosso parametro estimado \(\hat{\theta} \) tende a \(\theta \). Vamos
achar sua esperança.
\begin{align}
    \label{eq:Esperanca do score}
    E_{\theta }S(\theta)&=\int_{-\infty}^{\infty} p_{\theta } (x)S(\theta ) \,\mathrm{d}x\\
    &=\int_{-\infty}^{\infty} \frac{\partial }{\partial \theta } \ln L(\theta )p_{\theta } (x) \,\mathrm{d}x\\
    &=\int_{-\infty}^{\infty} \frac{\frac{\partial }{\partial \theta } L(\theta )}{L(\theta )}p_{\theta } (x) \,\mathrm{d}x\\
    &=\int_{-\infty}^{\infty} \frac{\partial }{\partial \theta } L(\theta )\,\mathrm{d}x\\
    &=\frac{\partial }{\partial \theta }\int_{-\infty}^{\infty}  p_{\theta } (x)\,\mathrm{d}x\\
    &=0
\end{align}
Sua variancia, por sua vez é definida pela esperança da informação de fisher, \(\mathcal{I}\), dado
o valor real de \(\theta \).
\begin{equation}
    \mathcal{I} =E[\frac{\partial}{\partial \theta } \ln (L(\theta )^{2})]=var_{\theta} S(\theta) 
\end{equation}
\begin{equation}
    \mathcal{I}=\int_{-\infty}^{\infty} (\frac{\partial }{\partial \theta } \ln L(\theta))^{2}  L(\theta)\,\mathrm{d}x 
\end{equation}
Ela é util para todos os valores possiveis de \(\theta \), enquanto a
informação de fisher observada, dada pelas equações nos capitulos inicias so é util nas vizinhanças
de \(\hat{\theta} \). De forma mais simples, \(\mathcal{I} \) é uma função de \(\theta \) para todos
os valores admissiveis de \(\theta \), dando uma média, por se dizer, atraves de todos os set de
dados. Esse valor nos da um valor geral para a curvatura da função score. O valor observado de
fisher, nosso I comum, nos da apenas para um unico dataset. O valor esperado, nosso \(\mathcal{I} \)
nos diz quão dificil vai  ser estimar nosso parametro, valores maiores implicam maior facilidade de
estimação. Para um modelo de Cauchy, dado a densidade de probabilidade \(p(x)=[\pi (1+x-\theta )^{2}]^{-1}\)
\[
    I(\theta )=-\sum_{i} 2\frac{[(x_{i}-\theta)^{2}-1 ]}{[(x_{i}-\theta)^{2}+1 ]^{2} }
\]
\[
    \mathcal{I} =-2nE_{\theta }\frac{(X_1-\theta )^{2} -1}{[(X_{1}-\theta)^{2}+1 ]^{2}}=\frac{n}{2} 
\]
\(I(\theta ) \neq \mathcal{I}\), qual dos dois é melhor escolher? \(I(\theta )\) para \(n=N\) nos
diz uma probabilidade de \(I(\hat{\theta}  )\) cair dentro de um intervalo. \(\mathcal{I} \) nos diz
a curvatura média. Qual é mais relevante? Posteriormente retornarei a esse problema. \par

Voltando a ideia da desigualdade de Cramer-Rao. Demonstrado que ela é valida no caso de
\(E(\hat{\theta})=\theta\), mas se formos para um caso mais geral, onde
\(E(\hat{\theta})=g(\theta)\), como nossa desigualdade fica? Bom, como tivemos que derivar e elevar
ao quadrado para chegarmos na desigualdade, sendo \(g(\theta )\) diferenciavel, teremos
\begin{equation}\label{eq:Cramer-Rao para caso gerais}
    Var(\theta ) \cdot E(\frac{\partial}{\partial \theta } \ln (L(\theta ))^{2} )\geq (g^\prime (\theta ))^{2} 
\end{equation}
\subsection{Uma analise sobre as diferenças entre \(I\) e \(\mathcal{I} \)}

Como foi colocado no capitulo anterior, existe uma diferença entre esses dois valores e irei me
debruçar um pouco mais sobre suas diferenças. \(I\) é definido como \(-\frac{\partial^{2}}{\partial \theta^{2} }\ln L(\theta) \)  
Em termos matematicos, analisando seu significado geometrico e sua derivada. Como isso, com valores
para todos as variaveis, nos da um escalar, essa informação nos da o valor da curvatura da função
score em um determinado \(\theta \). Valores de pico, muito altos, nos dizem que se tem uma menor
incerteza quanto ao parametro, valores proximos a zero nos dizem que há uma grande incerteza quanto
ao valor, indicando maior incerteza. Ele é estimado para o valor de maxima verossimilhança (MLE). O
MLE, na teoria de estimativa, nos conta uma estimativa no qual se tem maior chance do nosso
parametro está. Para perguntas do tipo, com um conjunto de dados, qual a melhor estimativa,
geralmente o MLE nós da uma resposta bem decente. \(I\) geralmente, vem junto com o estimador
\(\hat{\theta} \) então se é util acha-lo. Para isso, podemos resolver a equação \(S(\theta )=0\),
lembrando que \(S(\theta)=\frac{\partial }{\partial \theta } \ln (L(\theta))\). Onde achar essa
solução nos da o MLE, ou seja, o \(\hat{\theta} \). De forma mais importante, essa informação,
dependente do estimador, varia de dataset a dataset. Pensando numa normal, por exemplo, se fossemos
tirar 100 dados, 5 vezes da mesma distribuição, a probabilidade da média ser igual é praticamente
zero. Então, o valor de \(I\) seria diferente para cada uma delas. So seria uma util uma analise nas
redondezas de \(\hat{\theta} \). Essa informação de fisher é local e sensivel a mundaças de
coordenadas. \par

Por sua vez \(\mathcal{I} \) é definida como a varianca da função score, ou em outros termos,
\(E_{\theta }(\frac{\partial }{\partial \theta } L(\theta ))^{2}\), ou, \(-E_{\theta }(\frac{\partial^{2}  }{\partial \theta^{2}  } L(\theta ))\).
Repare nas diferenças, \(\mathcal{I} \) é o valor esperado da informação de fisher. \(\mathcal{I} \)
é a esperança de \(I\). Então, se \(I\) era a curvatura da função score em um determinado
\(\hat{\theta} \), \(\mathcal{I}\) vai nos dizer a curvatura média da função. Ademais, ele apresenta
algumas propriedades mais interessantes. Ele é invariante quanto a mudanças de coordenadas, ou seja,
se nossa função \(p(y|\theta )\) puder ser descrita como \(p_X(y-\theta)\), ambas nos darão o mesmo
valor de \(\mathcal{I} \). Para uma transformação de parametros, tal que \(\phi =g(\theta)\) para
qualquer distribuição \(g()\). A função score de \(\phi \) é dada por.
\begin{align}
    S(\phi)&=\frac{\partial }{\partial \phi }\ln  L(\theta )\\
    &=\frac{\partial \theta }{\partial \phi } \ln  L(\theta )\\
    &=\frac{\partial \theta }{\partial \phi }S(\theta)
\end{align}
E sua informação de fisher \(\mathcal{I} \) é
\begin{align}
    \mathcal{I} (\phi )&=varS(\phi )\\
    &=(\frac{\partial \theta  }{\partial \phi  })^{2} \mathcal{I}(\theta)\\
    &=\frac{\mathcal{I}(\theta)}{(\partial \phi /\partial \theta )^{2}}
\end{align}
Para uma poisson:
\begin{align*}
    \mathcal{I} (\theta)=\frac{n}{\theta  }\\
    \mathcal{I} (\ln \theta )=\frac{n/\theta }{1/\theta^{2} }=\theta
\end{align*}

\subsubsection{Matrizes de informação de Fisher}

Até dito momento so analisamos a informação de fisher para um parametro \(\theta \), ou seja, em uma
normal, assumimos que conheciamos a variancia \(\sigma^{2} \), mas, se por exemplo, não conhecermos
nem a media \(\mu \) nem a variancia \(\sigma^{2} \) como podemos descrever a informação observada e
esperada de fisher? Bom, para isso temos nossa matriz, a qual podemos colocar nossas informações
nela. A informação de fisher observada, na sua forma matricial pode ser dada por: \par
Seja \(\vec{\theta} \) o vetor coluna que contem todos os parametros variando de 1 a n. Para a \(i\)
observação, a informação de fisher pode ser descrita como
\begin{equation}\label{eq:informacao de fisher observada matricial}
    F_i(\vec{\theta})=(\frac{\mathrm{d}}{\mathrm{d}\vec{\theta} } \ln L(\vec{\theta}))
    (\frac{\mathrm{d}}{\mathrm{d}\vec{\theta} } \ln L(\vec{\theta})^{\top})
\end{equation}
Onde \(\frac{\mathrm{d}}{\mathrm{d}\vec{\theta} } \ln L(\vec{\theta})\) é um vetor coluna \(n \
\mathsf{x} \ 1\), onde nossa matriz de informação sera \(n \ \mathsf{x} \ n\). Para a informação
esperada, basta apenas fazer a esperança da matriz.
\begin{equation}\label{eq:informacao de fisher esperada matricial}
    F_i(\vec{\theta})=E[(\frac{\mathrm{d}}{\mathrm{d}\vec{\theta} } \ln L(\vec{\theta}))
    (\frac{\mathrm{d}}{\mathrm{d}\vec{\theta} } \ln L(\vec{\theta})^{\top})]
\end{equation}
Para o caso de derivadas de segunda ordem. Temos \par
Para a observação \(m,n\) temos:
\begin{equation}\label{eq:informacao de fisher observada matricial segunda ordem}
    F_i(\vec{\theta})=-\frac{\partial^{2} }{\partial \theta_m \partial\theta_j} \ln L(\theta )
\end{equation}
Similarmente para a esperada
\begin{equation}\label{eq:informacao de fisher esperada matricial segunda ordem}
    F_i(\vec{\theta})=-E(\frac{\partial^{2} }{\partial \theta_m \partial\theta_j} \ln L(\theta))
\end{equation}
\par
Pensando agora em uma limitação dela, se fossemos pegar a definição integral da informação de
fisher, \(\int_{-\infty}^{\infty} (\frac{\partial }{\partial \theta } \ln L(\theta ))^{2}*L(\theta )  \,\mathrm{d}x \),
\(\frac{\partial }{\partial \theta }L(\theta )= \frac{f^\prime (x|\theta )}{f(x|\theta)}\), essa função ela é indefinida se
\(f(x|\theta) \to 0 \), o que necessariamente ocorre dado um certo \(x\). Para contornarmos essa
situação podemos fazer uma substituição \(p(x)=g^{2}(x)\), de forma que nossa nova equação fica:
\begin{align}
    \mathcal{I} &=\int_{-\infty}^{\infty} (\frac{\partial }{\partial \theta } \ln L(\theta ))^{2}  \,\mathrm{d}x\\
    &=\int_{-\infty}^{\infty} (\frac{f^\prime (x) }{f(x)})^{2} f(x)  \,\mathrm{d}x\\
    &=\int_{-\infty}^{\infty} \frac{(2g(x))^{2} g^\prime(x)^{2} }{g(x)^{2}}  \,\mathrm{d}x\\
    &=4\int_{-\infty}^{\infty} g^\prime (x)^{2}   \,\mathrm{d}x 
\end{align}
Essa nova forma é a forma de amplitude da nossa informação. Ela então evitaria esse problema da não
existencia se \(p(x)\to 0\). Ademais, talvez não se aplicaria muito ao nosso caso, mas a integral e
o gradiente tem de existir, sendo uma condição necessaria essa.